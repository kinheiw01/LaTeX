% Packages

\documentclass{beamer}
\setcounter{tocdepth}{1}
\usepackage{graphicx}
\usepackage[export]{adjustbox}
\graphicspath{{Images/}{./}} % Specifies where to look for included images (trailing slash required)
\usepackage{tikz}

\usepackage{booktabs} % Allows the use of \toprule, \midrule and \bottomrule for better rules in tables
\usepackage{pgfplots}
\usepackage{tikz}
\pgfplotsset{width=6cm, compat=newest, every tick label/.append style={scale=0.5}}

% Theme
\usetheme{Madrid}

% Font
\usefonttheme{serif}
\usepackage{newtxtext,newtxmath}
\usepackage[default]{lato}

% Inner theme
\useinnertheme{circles}

% Information
\title{Additional algebra}
\author{Kin Hei Wong}
\date{\today}
%%%%%%%%%%%%%%%%%%%%%%%%%%%%%%%%%%%%%%%%%%%%%%%%%%%%%%%%%%
\begin{document}

% Title slide
\begin{frame}
    \titlepage
\end{frame}

% Table of Content
\begin{frame}
    \frametitle{Presentation overview}
    \tableofcontents
\end{frame}
%%%%%%%%%%%%%%%%%%%%%%%%%%%%%%%%%%%%%%%%%%%%%%%%%%%%%%%%%%%
% Body slides

\section{4A: Polynomial identities}
\begin{frame}
    \frametitle{4A}
    \begin{center}
        \title{Polynomial identities}
        \maketitle
    \end{center}
\end{frame}

\begin{frame}{What is a polynomial?}
    Polynomial function is written with the form of:\\
    $P(x) = a_nx^n + a{n-1}x^{n-1} + \dots + a_1x + a_0$\\
    where $n$ is the natural number or zero, and the coefficients $a_0, \dots, a_n$ are real numbers with $a_n \neq 0$.\\
    Leading term = $a_nx^n$\\
    Degree of polynomial = $n$\\
    Monic polynomial = Polynomial whose leading term has coefficient 1\\
    Constant term = Term of index 0.\\
    Note: $P(x) = 0$ is \textbf{zero polynomial}; degree is undefined
\end{frame}

\begin{frame}[t]
    \frametitle{Example 1}
    If $(a + 2b)x^2  - (a - b)x + 8 = 3x^2 - 6x + 8$ for all $x$, find the values of $a$ and $b$.
\end{frame}

\begin{frame}[t]
    \frametitle{Example 2}
    Express $x^2$ in the form $c(x-3)^2 + a(x-3) + d$.
\end{frame}

\begin{frame}[t]
    \frametitle{Example 3}
    Find the values of $a,b,c$ and $d$ such that\\
    $x^3 = a(x+2)^3 + b(x+1)^2 + cx + d$ for all $x$
\end{frame}
\begin{frame}
\end{frame}

\begin{frame}[t]
    \frametitle{Example 4}
    Show that $2x^3 - 5x^2 + 4x + 1$ cannot be expressed in the form $a(x+b)^3 + c$.
\end{frame}
\begin{frame}
\end{frame}

\begin{frame}{Exercise 4A}
\end{frame}
%%%%%%%%%%%%%%%%%%%%%%%%%%%%%%%%%%%%%%%%%%%%%%%%%%%%%%%%%%%%%%%
\section{4B: Quadratic equations}
\begin{frame} 
    \frametitle{4B}
    \begin{center}
        \title{Quadratic equations}
        \maketitle
    \end{center}
\end{frame}

\begin{frame}{Polynomial (Quadratic)}
    We say quadratic function is known as a polynomial function of degree 2. Hence, the general quadratic function can be written as:\\
    $P(x) = ax^2 + bx + c$, where $a \neq 0$, $P(x)\in \mathcal{P}^2$\\
    \bigskip
    To solve $0 = ax^2 + bx + c$, we can factorise, completing the square, or the general formula below (which is derived from completing the square)\\
    \begin{block}{General quadratic formula}
        $x = \frac{-b \pm \sqrt{b^2 - 4ac}}{2a}$
    \end{block}
\end{frame}

\begin{frame}[t]
    \frametitle{Example 5}
    Solve the following quadratic equations for $x$:
    \begin{enumerate}
        \item $2x^2 + 5x = 12$
        \item $3x^2 + 4x = 2$
        \item $9x^2 + 6x + 1 = 0$
    \end{enumerate}    
\end{frame}
\begin{frame}
\end{frame}

\begin{frame}{What is discriminant?}
    Discriminant is used to determine the number of solutions to a quadratic. It is within the general formula.\\
    $b^2 - 4ac = \Delta$\\
    \begin{block}{Real solution}
        \begin{enumerate}
            \item If $\Delta >0$, there are two real solutions
            \item If $\Delta =0$, there are one real solutions
            \item If $\Delta <0$, there are no real solutions (but two complex solutions)
        \end{enumerate}
    \end{block}
    \begin{block}{Rational solution}
        If $a,b,c \in \mathbb{Q}$\\
        \begin{enumerate}
            \item If $\Delta \neq 0$ AND $\Delta$ is perfect square, it has two rational solutions
            \item If $\Delta = 0$, then one rational solution
            \item If $\Delta > 0$ AND $\Delta$ is not a perfect square, it has two irrational solutions.
        \end{enumerate}
    \end{block}
\end{frame}

\begin{frame}[t]
    \frametitle{Example 6}
    Consider the quadratic equation $x^2-4x = t$. Make $x$ the subject and give the values of $t$ for which real solution(s) to the equation can be found.
\end{frame}

\begin{frame}[t]
    \frametitle{Example 7}
    \begin{enumerate}
        \item Find the discriminant of quadratic $x^2 + px - \frac{25}{4}$ in terms of $p$.
        \item Solve the quadratic equation $x^2 + px - \frac{25}{4} = 0$ in terms of $p$.
        \item Prove that there are two solutions for all values of $p$.
        \item Find the values of $p$, where $p$ is a non-negative integer, for which the quadratic equation has rational solutions.
    \end{enumerate}
\end{frame}

\begin{frame}[t]
    \frametitle{Example 8}
    A rectangle has an area of $288 cm^2$. If the width is decreased by $1cm$ and the length increased by $1cm$, the area would be decreased by
    $3cm^2$. Find the original dimensions of the rectangle.
    
\end{frame}

\begin{frame}[t]
    \frametitle{Example 9}
    Solve the equation $x - 4\sqrt{x} - 12 = 0$ for $x$ 
\end{frame}

\begin{frame}{Exercise 4B}
\end{frame}
%%%%%%%%%%%%%%%%%%%%%%%%%%%%%%%%%%%%%%%%%%%%%%%%%%%%%%%%%%%%%%%
\section{4C: Applying quadratic equations to rate problems}
\begin{frame} 
    \frametitle{4C}
    \begin{center}
        \title{Applying quadratic equations to rate problems}
        \maketitle
    \end{center}
\end{frame}

\begin{frame}{Rate}
    Before going to the application of calculation, we must understand what rate is.\\
    A rate describes how a certain quantity changes with respect to the change in another quantity (often time).\\
    Examples of rate: Speed, flow, acceleration\\
    E.g. In speed, it is how fast it is to travel from start to end\\
    If time is higher, then the speed(rate) is lower, else vice versa
\end{frame}

\begin{frame}[t]
    \frametitle{Example 10}
    \begin{enumerate}
        \item Express $\frac{6}{x} + \frac{6}{x+8}$ as a single fraction.
        \item Solve the equation $\frac{6}{x} + \frac{6}{x+8} = 2$ for $x$
    \end{enumerate}
\end{frame}

\begin{frame}[t]
    \frametitle{Example 11}
    A car travels 500$km$ at a constant speed. If it had travelled at a speed of 10$km/h$ less, it would have taken 1 hour 
    more to travel the distance. Find the speed the car.
\end{frame}

\begin{frame}[t]
    \frametitle{Example 12}
    A tank is filled by two pipes. The smaller pipe alone will take 24 minutes longer than the larger pipe alone, and 32 minutes longer than 
    when both pipes are used. How long will each pipe take to fill the tank alone? How long will it take for both pipes used together to fill the tank?
\end{frame}
\begin{frame}
\end{frame}

\begin{frame}{Exercise 4C}
\end{frame}
%%%%%%%%%%%%%%%%%%%%%%%%%%%%%%%%%%%%%%%%%%%%%%%%%%%%%%%%%%%%%%%
\section{4D: Partial fractions}
\begin{frame} 
    \frametitle{4D}
    \begin{center}
        \title{Partial fractions}
        \maketitle
    \end{center}
\end{frame}

\begin{frame}{Partial fractions}
    A rational function is the quotient of two polynomials. If $P(x)$ and $Q(x)$ are polynomials, then $f(x) = \frac{P(x)}{Q(x)}$ is a rational function\\
    If the degree of $P(x)$ is greater than $Q(x)$, it is known as improper fraction, else it is proper fraction.
\end{frame}

\begin{frame}[t]
    \frametitle{Example 13}
    Resolve $\frac{3x + 5}{(x-1)(x+3)}$ into partial fractions.
\end{frame}

\begin{frame}[t]
    \frametitle{Example 14}
    Resolve $\frac{2x + 10}{(x+1)(x-1)^2}$ into partial fractions.
\end{frame}
\begin{frame}
\end{frame}

\begin{frame}[t]
    \frametitle{Example 15}
    Resolve $\frac{x^2 + 6x + 5}{(x-2)(x^2 + x + 1)}$ into partial fractions.
\end{frame}
\begin{frame}
\end{frame}

\begin{frame}[t]
    \frametitle{Example 16}
    Resolve $\frac{x^5 + 2}{x^2-1}$ into partial fractions.
\end{frame}
\begin{frame}
\end{frame}

\begin{frame}{Exercise 4D}
\end{frame}
%%%%%%%%%%%%%%%%%%%%%%%%%%%%%%%%%%%%%%%%%%%%%%%%%%%%%%%%%%%%%%%
\section{4E: Simultaneous equations - Advanced}
\begin{frame} 
    \frametitle{4E}
    \begin{center}
        \title{Simultaneous equations - Advanced}
        \maketitle
    \end{center}
\end{frame}

\begin{frame}[t]
    \frametitle{Example 17}
    Find the coordinates of the points of intersection of the parabola with equation $y = x^2 -2x - 2$ and the straight line with equation 
    $y = x+ 4$
\end{frame}

\begin{frame}[t]
    \frametitle{Example 18}
    Find the points of intersection of the circle with equation $(x-4)^2 + y^2 = 16$ and the line with equation $x-y=0$.
\end{frame}

\begin{frame}[t]
    \frametitle{Example 19}
    Find the points of contact of the straight line with equation $\frac{1}{9}x+y = \frac{2}{3}$ and the curve with equation $xy = 1$
\end{frame}

\begin{frame}[t]
    \frametitle{Example 20}
    Find the coordinates of the points of intersection of the graphs of $y=-3x^2 -4x + 1$ and $y=2x^2 - x -1$.
\end{frame}
\begin{frame}{Exercise 4E}
\end{frame}
%%%%%%%%%%%%%%%%%%%%%%%%%%%%%%%%%%%%%%%%%%%%%%%%%%%%%%%%%%%%%%%

\end{document}